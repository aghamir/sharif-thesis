%\documentclass[MScThesis,twoside]{sharifthesis}
%\documentclass[MScThesis,oneside]{sharifthesis}
\documentclass[PhDThesis,twoside]{sharifthesis}
%\documentclass[PhDThesis,oneside]{sharifthesis}
%\documentclass[PhDProposal,twoside]{sharifthesis}
%\documentclass[PhDProposal,oneside]{sharifthesis}

%\def\enSubject{Master of Science Thesis (Information Technology Major), Computer Engineering Department, Sharif University of Technology, Tehran, I. R. Iran}
\def\enSubject{Doctor of Philosophy Thesis (Information Technology Major), Computer Engineering Department, Sharif University of Technology, Tehran, I. R. Iran}
%\def\enSubject{Doctor of Philosophy Research Proposal (Information Technology Major), Computer Engineering Department, Sharif University of Technology, Tehran, I. R. Iran}

\def\enTitle{Title of thesis (prefer one-line titles)}
\def\enAuthor{‌Behnam Momeni}
\def\enKeywords{First Key Word, Second Key Word, Final Key Word}


\usepackage{amssymb}
\usepackage{algorithmicx}
\usepackage{graphicx}
\usepackage{array}
\usepackage{multirow}
\usepackage{comment}
\usepackage[style=ieee,backend=biber]{biblatex}
\newcommand{\bibliographytitle}{\rl{کتاب‌نامه}}
\usepackage{fancybox}
\usepackage{paralist}
\usepackage{textcomp}
\usepackage{rotating}

\newcounter{tablerow}
\renewcommand{\arraystretch}{1.5}
\usepackage{cleveref}
\newcommand{\crefrangeconjunction}{--}
\crefformat{tablerow}{#2#1#3}
\crefmultiformat{tablerow}{#2#1#3}%
{ و~#2#1#3}%
{, #2#1#3}%
{ و~#2#1#3}
\crefformat{section}{#2#1#3}
\crefmultiformat{section}{#2#1#3}%
{ و~#2#1#3}%
{, #2#1#3}%
{ و~#2#1#3}



\newcommand{\URL}%
{یو.آر.ال.}
\newcommand{\postgresql}%
{پُستگِرِس.کیو.اِل.}
\newcommand{\booktabs}%
{بوک‌تَبز}
\newcommand{\ctablePersian}%
{سی‌تِیبِل}

\نوواژه[کلید]{نمونه}{Example}
\نوواژه{خودگرد}{Automata}
\نوواژه{قالب}{Template}
\نوواژه{پوشه}{Folder}
\نوواژه{پرونده}{File}
\نوواژه{بسته}{Package}



\newcommand{\myrotate}[3][]{\rotatebox{90}{\parbox[c][#1]{#2}{\centering\arraybackslash\rl{#3}}}}
\newcommand{\multilinescell}[2][c]{\begin{tabular}[#1]{@{}c@{}}#2\end{tabular}}
\newcommand{\twolinescell}[3][c]{\multilinescell[#1]{#2\\#3}}
\newcommand{\itemrl}[1][]{\item[\rl{#1}]}
\eqcommand{چرخش}{myrotate}
\eqcommand{سلولچندخطی}{multilinescell}
\eqcommand{سلول‌دوخطی}{twolinescell}
\eqcommand{فقره‌راست}{itemrl}



\newcommand{\Eqn}[1]%
{فرمول~(#1) }
\newcommand{\Eqns}[1]%
{فرمول‌های~(#1) }

\eqcommand{فرمول}{Eqn}
\eqcommand{فرمولهای}{Eqns}

\newcolumntype{C}[1]{>{\centering\arraybackslash}m{#1}}


\graphicspath{{img/}}% tell tex engine address of folder containing your pictures


\hypersetup{
	pdftitle = {\enTitle},
	pdfauthor = {\enAuthor},
	pdfsubject = {\enSubject},
	pdfkeywords = {\enKeywords}
}

\addbibresource{resources/resources.bib}


\newcommand{\faKeywords}{واژه‌ی کلیدی نخست،
واژه‌ی کلیدی دوم،
واژه‌ی کلیدی پایانی}
\eqcommand{واژه‌های‌کلیدی}{faKeywords}
\آرم{\درج‌تصویر[scale=.7]{logo}}
\تاریخ{اردیبهشت ۱۳۹۳}
\عنوان{عنوان پایان‌نامه (ترجیح بر تک‌خط بودن)}
\نویسنده{بهنام مومنی}
\دانشگاه{{\نستعلیق\درشت‌تر دانشگاه صنعتی شریف %
\\[0.6cm]}
دانشکده‌ی مهندسی کامپیوتر}
\دانشگاه‌عادی{دانشگاه صنعتی شریف\\
دانشکده‌ی مهندسی کامپیوتر}
\موضوع{گرایش به زبان پارسی}
\استادراهنما{دکتر <نام استاد راهنما>}
%اگر استاد مشاور ندارید، خط زیر را comment کنید
%همچنین، فراموش نکنید در آخر این پرونده، اطلاعات انگلیسی معادل این دستورها را هم پر کنید
\استادمشاور{دکتر <نام استاد مشاور>} 

\newcommand{\efootnote}[1]{\footnote{\lr{#1}}}
\newcommand{\ecfootnote}[1]{}


% ================ Correct hyphenations ================
\hyphenation{test}


\makeglossaries
%\includeonly{related_works/related_works}
%\includeonly{evaluation/evaluation}

% ===== DEPRACATED AREA =====
% Following commands are provided to make older documents compilable.
% These commands are depracated and should not be used in new documents.
\newcommand{\ترجمه‌ج}[2]
{\ترجمه[#1‌ها]{#1}{#2}}
\newcommand{\ترجمه‌جمع}[3]
{\ترجمه[#3]{#1}{#2}}
\newcommand{\برگردان}[3]
{\ترجمه{#1}{#3}\زیرنویس{#2}}
\eqcommand{اسم}
{نام}
% ===== END OF DEPRACATED AREA =====

\شروع{نوشتار}


\newcommand{\StartDocument}{\frontmatter \baselineskip1.2\baselineskip \pagestyle{empty} \null \vfill
\شروع{وسط‌چین}
{\نستعلیق‌درشت بسم اللّه الرحمن الرحیم}
\پایان{وسط‌چین}
\vfill}

%the initial title is supposed to be printed on the cover.
%for non final version, you can leave following commands as is to create only one title page (printed on paper)
%for final version you need to swap folowing commented/uncommented makethesistitle commands to achieve this order: title on the cover THEN in the name of god page THEN another title page but this time printed on paper
\makethesistitle
\StartDocument
%\makethesistitle

\pagestyle{pagenumberonlyPagestyle}
% following parts are not required in PhD proposal and should be removed. BEGIN OF COMMENT FOR PhD Proposal........
%صفحه‌ی تصویب در پیشنهاد پژوهشی وجود ندارد.
\شروع{تصویب}
%خط‌های زیر در صورت نبود استاد مدعو comment شوند
\داور{استاد مدعو}{دکتر <نام استاد مدعو ۱>}
\داور{استاد مدعو}{دکتر <نام استاد مدعو ۲>}
\پایان{تصویب}
\newpage

\تقدیم{\درشت تقدیم به ...؛   صفحه‌ی تقدیم اختیاری است.} 

\setlength{\baselineskip}{0.9cm}
\begin{comment}
\فصل*{پیش‌گفتار}
\thispagestyle{pagenumberonlyPagestyle}
پیشگفتار اختیاری است. در صورت تمایل به نگارش پیش‌گفتار، محیط کامنت که آن را دربرگرفته باید حذف شود.
\end{comment}

\شروع{قدردانی}
صفحه‌ی قدردانی. این صفحه اختیاری بوده و می‌توانید آن را حذف کنید. برای این کار کافی است محیط قدردانی در پرونده‌ی تِک را حذف کنید. متداول است که در این صفحه از خانواده، استادها و همکارهای خود قدردانی نمایید.
\پایان{قدردانی}
% END OF COMMENT FOR PhD Proposal.

\شروع{چکیده}{\واژه‌های‌کلیدی}
% abstract ...
% write it at the end...
%persian abstract

چکیده‌ی پایان‌نامه به زبان پارسی را پس از نگارش کامل پایان‌نامه آماده کنید. چکیده از $300$ واژه (یا کمتر) تشکیل شده و در ادامه‌ی آن $4$ تا $7$ واژه‌ی کلیدی بیان می‌شود. واژه‌های کلیدی در پرونده‌ی اصلی (به زبان پارسی و انگلیسی) نوشته می‌شوند و چکیده بسته به زبان در دو پرونده‌ی جداگانه در پوشه‌ی عمومی نوشته می‌شود.

\پایان{چکیده}


\setlength{\baselineskip}{0.9cm}
\pagenumbering{tartibi}\tableofcontents\listoftables\listoffigures
%list of abbreviations may be added here...


\PrepareForMainContent
\input{general/thesis_content}



\PrepareForBibliography

\setlatintextfont[Scale=1]{Linux Libertine}
\setlength{\baselineskip}{0.8cm}
%\setromantextfont[Scale=1.2]{XB Niloofar}

%\bibliographystyle{IEEEtran}
%\bibliographystyle{is-unsrt}
%\bibliographystyle{ieeetr-fa}
%\bibliographystyle{amsplain}

%\bibliography{resources/resources}
\latin
\printbibliography[title=\bibliographytitle,heading=bibintoc]
\persian

% glossaries
{\cleardoublepage\setlength{\baselineskip}{1cm}\printpersianglossary\cleardoublepage\printenglishglossary}


\PrepareForLatinPages
\date{April 2014}
\logo{\includegraphics[scale=.4]{logo-en}}
\title{\sffamily\enTitle}
\author{\sffamily\enAuthor}
\university{\normalfont\bfseries Sharif University of Technology\\Computer Engineering Department}
\subject{Your Major in English Language}
\supervisor{\sffamily Dr. <name of advisor prof.>}
%If you don't have a consultant professor, comment following line
\consult{\sffamily Dr. <name of consultant prof.>}
\begin{abstract}{\enKeywords}
\input{general/abstract-en}
\end{abstract}
\makethesistitle
\پایان{نوشتار}
