\قسمت{شیوه‌ی نگارش پایان‌نامه}\برچسب{sec:intro:writing}

این بخش به شیوه‌ی نگارش پایان‌نامه و چگونگی به‌کارگیری این \واژه{قالب} اختصاص دارد.
در این قالب، هر فصل از پایان‌نامه با یک \واژه{پوشه} مشخص می‌شود. برای نمونه، فصل مقدمه که در حال خواندن آن هستید، در پوشه‌ای با نام \نام{اینتروداکشِن}{introduction} نوشته شده است. در هر پوشه، یک \واژه{پرونده} با همان نام قرار دارد که متن آغازین فصل در آن نوشته می‌شود. این توصیف باید هدف از نگارش آن فصل را بیان کرده، به همه‌ی بخش‌های آن ارجاع نموده و در انتها پرونده‌های جداگانه‌ای را که به هر بخش (در همان پوشه) اختصاص پیدا کرده‌اند، با دستور \نام{اینپوت}{input} دربرگیرد.

انگیزه‌ی اصلی از تقسیم‌بندی فصل‌ها و بخش‌ها (به ترتیب) در پوشه‌ها و پرونده‌ها، این است که مدیریت حجم انبوهی از نوشته‌ها در یک پایان‌نامه، با این تقسیم‌بندی ساده‌تر خواهد شد. همچنین توصیه می‌شود از \ترجمه[سامانه‌ی پایش  نسخه‌های]{سامانه‌ی پایش  نسخه‌ها}{Version Control System} \نام{گیت}{Git} برای دنبال کردن تغییرها در متن پایان‌نامه بهره بگیرید.

اگر به متن \نام{تِک}{\TeX} این نوشته تا به اینجا نگاه کنید، چند دستور کلیدی را مشاهده می‌کنید که برای ترجمه‌ی واژه‌ها به‌کار رفته‌اند. شایان ذکر است که در همه‌ی متن پایان‌نامه، نباید از الفبای غیر پارسی (از جمله انگلیسی) بهره گرفته شود. در صورتی که می‌خواهید واژه‌ی بیگانه‌ای را به‌کار بگیرید و معادل پارسی برای آن موجود نیست، لازم است که آن واژه را با الفبای پارسی نوشته و آنگاه از زیرنویس برای بیان شیوه‌ی نگارش آن در زبان انگلیسی (برای نمونه \نام{اِسمیت}{Smith}) بهره  بگیرید. در ادامه ۴ دستور کلیدی در این رابطه معرفی می‌شوند.

\شروع{شمارش}
	\باره دستور \موکد{ترجمه} با یک آرگومان اختیاری و دو آرگومان اجباری فراخوانی می‌شود. دو آرگومان اجباری به ترتیب واژه‌ی پارسی و انگلیسی را مشخص می‌کنند. واژه‌ی پارسی در متن نوشته شده، واژه‌ی انگلیسی در زیرنویس آورده شده و دو سطر به واژه‌نامه‌ها در انتهای پایان‌نامه افزوده می‌گردد (یک سطر به واژه‌نامه‌ی پارسی به انگلیسی و یک سطر به واژه‌نامه‌ی انگلیسی به پارسی). ممکن است واژه‌ای که می‌خواهید در متن نوشته شود با واژه‌ای که می‌خواهید در واژه‌نامه آورده شود، متفاوت باشد. برای نمونه می‌خواهید واژه‌ی \ترجمه[داده‌ساختارهای]{داده‌ساختار}{Data Structure} گوناگون را در متن بیاورید، ولی حالت مفرد آن، یعنی داده‌ساختار را به واژه‌نامه اضافه کنید. در این حالت می‌توانید آنچه را که در متن نوشته می‌شود، با آرگومان اختیاری دستور \موکد{ترجمه} تعیین کنید.
	
	\باره دستور \موکد{نوواژه} همانند دستور بالا دو آرگومان اجباری برای مشخص کردن واژه‌های پارسی و انگلیسی در دو واژه‌نامه دریافت می‌کند. با این تفاوت که هیچ واژه‌ای را در متن نمی‌نویسد و به طور خودکار سطری به واژه‌نامه‌ها نمی‌افزاید. بلکه همانند تعریف مرجع‌ها در \نام{بیبتِک}{BibTeX} تنها واژه‌ها را آماده‌ی به‌کارگیری می‌کند. اگر واژه توسط دستور \موکد{واژه} در متن به‌کارگرفته شد، آن را به واژه‌نامه‌ها می‌افزاید و در غیر این صورت هیچ کاری نمی‌کند. دستورهای \موکد{نوواژه} را در پرونده‌ی خاص واژه‌نامه\زیرنویس{این پرونده در پوشه‌ی general و با نام \چر{glossaries.tex} جای دارد. به چگونگی استفاده از زیرنویس به جای به‌کارگیری الفبای انگلیسی در متن و همچنین دستور \موکد{چر} توجه کنید.} بنویسید. این کار سبب می‌شود که بتوان واژه‌نامه و جای‌گزین‌های تعریف شده توسط شما را در آینده به اشتراک گذاشت. همچنین این دستور یک آرگومان اختیاری به عنوان کلید دریافت می‌کند. در صورت عدم تعیین کلید، واژه‌ی پارسی نقش کلید را ایفا می‌کند.
	
	\باره دستور \موکد{واژه} همراه با دستور بالا کار می‌کند. این دستور کلید مشخص شده در دستور \موکد{نوواژه} را به عنوان تنها آرگومان اجباری خود دریافت کرده و واژه‌ی پارسی مرتبط را نشان می‌دهد. اگر نخستین باری باشد که آن واژه در متن آورده شده است، انگلیسی آن هم در زیرنویس آورده می‌شود. اگر همانند حالت دستور \موکد{ترجمه} بخواهید گونه‌ای متفاوت از واژه در متن آورده شود، آن را به عنوان آرگومان اختیاری دستور \موکد{واژه} مشخص کنید. توصیه می‌شود به جای \موکد{ترجمه} از \موکد{نوواژه} و \موکد{واژه} استفاده شود، زیرا در صورت جابه‌جایی بخش‌ها، تشخیص نخستین کاربرد یک واژه، خودکار انجام می‌شود، ولی در \موکد{ترجمه} باید نخستین کاربرد با دستور \موکد{ترجمه} معلوم شود و کاربردهای بعدی به صورت معمولی (بدون دستور) نوشته شوند.
	
	\باره دستور \موکد{نام} دو آرگومان اجباری را دریافت کرده و آرگومان پارسی تخست را در متن و آرگومان انگلیسی دوم را در زیرنویس  می‌آورد. این دستور به واژه‌نامه، هیچ سطری را اضافه نمی‌کند. کاربرد این دستور برای نام‌های غیر پارسی است که به دلیل مجاز نبدون به‌کارگیری الفبای انگلیسی در متن پایان‌نامه باید در زیرنویس آورده شوند.
\پایان{شمارش}


\begin{figure}[tbp]
\centering
\includegraphics[width=0.7\linewidth]{ce_department}
\def\mytempcaption{تصویری از دانشکده‌ی مهندسی کامپیوتر در دانشگاه صنعتی شریف}
\caption[\mytempcaption]{\mytempcaption~\cite{sharif:cesite}}
\label{fig:cedepartment}
\end{figure}

در ادامه نمونه‌هایی از لیست غیر شمارشی، شکل و جدول آورده شده‌اند. برای نمایش شکل‌ها، کافی است پرونده‌ی تصویر مورد نظر را در پوشه‌ی تصویرها\زیرنویس{به نام img} قرار داده و بدون اشاره به پوشه‌ی دربرگیرنده‌ی آن، تنها نام تصویر را ذکر کنید. به هر شکل و جدول باید دست کم یک بار در متن ارجاع داده شود. برای نمونه شکل~\رجوع{fig:cedepartment} تصویری از دانشکده‌ی مهندسی کامپیوتر را نشان می‌دهد. ارجاع مناسب شکل‌ها باید در عنوان آن‌ها صورت گیرد. برای اینکه این ارجاع در فهرست شکل‌ها آورده نشود از آرگومان اختیاری مرتبط همچون نمونه‌ی آورده شده، استفاده کنید.


\begin{table}[tbp]
\شرح{چهار دستور کلیدی در ترجمه‌ی واژه‌ها}
\برچسب{tbl:translate}
\centering
\begin{tabular}{ccC{7cm}}
\toprule
%\hline \rowcolor[gray]{.7}
نام دستور & نمونه & شرح \\
\midrule
ترجمه & \textbackslash ترجمه[نمونه‌های]\{نمونه\}\{Example\} & آرگومان اختیاری یا واژه‌ی پارسی را در متن آورده، واژه‌ی انگلیسی را در زیرنویس آورده و دو نگاشت را به دو واژه‌نامه خواهد افزود. \\
نوواژه & \textbackslash نوواژه[schema]\{شِما\}\{Schema\} & آرگومان اختیاری کلید را مشخص می‌کند. دو آرگومان دیگر معلوم می‌کنند در صورت به‌کارگیری کلید در متن، چه سطری به واژه‌نامه‌ها افزوده شود. کلید پیش‌فرض واژه‌ی پارسی است. \\
واژه & \textbackslash واژه[شِماهایی]\{schema\} & آرگومان اختیاری تعیین می‌کند در متن چه واژه‌ای باید نوشته شود. کلید (که به طور پیش‌فرض در متن نوشته می‌شود)، سطر مرتبط از دستور نوواژه را نشان می‌دهد. \\
نام & \textbackslash نام\{\URL\}\{URL\} & آرگومان نخست در متن (نشان‌دهنده‌ی یک نام) و آرگومان دوم در زیرنویس آورده می‌شود. \\
\bottomrule
\end{tabular}
\end{table}

جدول~\رجوع{tbl:translate} برای نمونه، فهرستی از چهار دستور توصیف شده‌ی بالا را گردآوری کرده است.
برای جدول‌ها، همانند این نمونه از \واژه[بسته‌ی]{بسته} \booktabs{} بهره گرفته و از خط‌های عمودی یا رنگ کردن خانه‌ها در جدول پرهیز کنید.
برای آخرین نمونه یک لیست بدون ترتیب آورده شده است که در آن برخی از خطاهای رایج در نگارش متن‌های پارسی ذکر شده‌اند.

\شروع{بارهها}
	\باره حرف اضافه‌ی \موکد{را} هرگز نباید بعد از فعل آورده شود،
	
	\باره ترجیح بر نگارش واژ \موکد{ی} بزرگ به جای گونه‌ی کوچک آن است. برای نمونه خانه‌ی ما بر خانهٔ ما ترجیح دارد. دلیل این ترجیح اشتباه شدن گونه‌ی کوچک \موکد{ی} با واژ همزه است،
	
	\باره جمع مکسر، همزه و تنوین در پارسی وجود نداشته و باید با شکل‌های متناسب جای‌گزین شوند. می‌توانید به جای مثلاً بگویید برای نمونه، به جای مسئله بنویسید مساله (برای عدم نیاز به نگارش همزه)، به جای تغییرات بگویید تغییرها و به جای موارد بگویید موردها،
	
	\باره برای اینکه تشخیص دهید کدام واژه‌ها را باید جدا نوشت و کدام واژه‌ها را باید یکپارچه نوشت، از یک قاعده‌ی ساده پیروی نمایید:
	اگر معنی واژه‌ای از اجزای آن بدون کم و کاستی قابل برداشت است، آن را جدا (با نیم‌فاصله\زیرنویس{به نیم‌فاصله، فاصله‌ی مجازی هم می‌گویند.}) بنویسید و اگر باید آن واژه را به عنوان واژه‌ای مستقل به خاطر سپرد،
	یکپارچه (سرهم) بنویسید. برای نمونه \موکد{جوانمرد} باید یکپارچه نوشته شود، ولی \موکد{جای‌گزین} باید جدا نوشته شود،
	
	\باره اگر از \ترجمه{ سامانه‌ی عامل}{Operating System} \نام{ویندوز}{Windows} استفاده می‌کنید، صفحه‌کلید \ترجمه[استانده‌ی]{استانده}{Standard} ایران را از~\مرجع{persiancomputing:keyboard} دریافت کنید. اگر از  سامانه‌ی عامل \نام{لینوکس}{Linux} استفاده می‌کنید،   صفحه‌کلید استانده‌ی ایران  به طور پیش‌فرض پشتیبانی می‌شود.
	
\پایان{بارهها}

در پایان هر بخش یا فصل مناسب است در جمله‌ای کوتاه به بخش یا فصل بعدی اشاره کنید. در بخش بعدی ساختار این پایان‌نامه بیان می‌گردد.
