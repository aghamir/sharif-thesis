\usepackage{amssymb}
\usepackage{mathrsfs}
\usepackage{algorithmicx}
\usepackage{graphicx}
\usepackage{multirow}
\usepackage{comment}
\usepackage[style=ieee,backend=biber]{biblatex}
\newcommand{\bibliographytitle}{\rl{کتاب‌نامه}}
\usepackage{paralist}
\usepackage{textcomp}
\usepackage{ctable}% for tables (it provides table-notes and imports booktabs package too)

\newcounter{tablerow}
\renewcommand{\arraystretch}{1.5}
\usepackage{cleveref}
\newcommand{\crefrangeconjunction}{--}
\crefformat{tablerow}{#2#1#3}
\crefmultiformat{tablerow}{#2#1#3}%
{ و~#2#1#3}%
{, #2#1#3}%
{ و~#2#1#3}
\crefformat{section}{#2#1#3}
\crefmultiformat{section}{#2#1#3}%
{ و~#2#1#3}%
{, #2#1#3}%
{ و~#2#1#3}



\newcommand{\URL}%
{یو.آر.ال.}
\newcommand{\postgresql}%
{پُستگِرِس.کیو.اِل.}
\newcommand{\booktabs}%
{بوک‌تَبز}
\newcommand{\ctablePersian}%
{سی‌تِیبِل}

\نوواژه[کلید]{نمونه}{Example}
\نوواژه{خودگرد}{Automata}
\نوواژه{قالب}{Template}
\نوواژه{پوشه}{Folder}
\نوواژه{پرونده}{File}
\نوواژه{بسته}{Package}



\newcommand{\myrotate}[3][]{\rotatebox{90}{\parbox[c][#1]{#2}{\centering\arraybackslash\rl{#3}}}}
\newcommand{\multilinescell}[2][c]{\begin{tabular}[#1]{@{}c@{}}#2\end{tabular}}
\newcommand{\twolinescell}[3][c]{\multilinescell[#1]{#2\\#3}}
\newcommand{\itemrl}[1][]{\item[\rl{#1}]}
\eqcommand{چرخش}{myrotate}
\eqcommand{سلولچندخطی}{multilinescell}
\eqcommand{سلول‌دوخطی}{twolinescell}
\eqcommand{فقره‌راست}{itemrl}



\newcommand{\Eqn}[1]%
{فرمول~(#1) }
\newcommand{\Eqns}[1]%
{فرمول‌های~(#1) }

\eqcommand{فرمول}{Eqn}
\eqcommand{فرمولهای}{Eqns}

\فرمان‌نو{\نگا}{ن.بـ.}
\فرمان‌نو{\نگاص}[1]{\نگا{} صفحه‌ی~\رجوع‌صفحه{#1}}
\فرمان‌نو{\آیم}[1][$i$]%
{#1اُم}

\newcolumntype{C}[1]{>{\centering\arraybackslash}m{#1}}

\فرمان‌نو{\نامک}[1]{%
\fcolorbox{red}{yellow}{\begin{minipage}{\textwidth}#1\end{minipage}}}

\newcommand{\tobewritten}[1]{\نامک{%
این جعبه، باید با متن متناظر با \چر{#1} جای‌گزین گردد.

برای دریافتن معنی \چر{#1} به پرونده‌ی \چر{TODO} نگاه کنید.
}}

\graphicspath{{img/}}% tell tex engine address of folder containing your pictures


\hypersetup{
	pdftitle = {\enTitle},
	pdfauthor = {\enAuthor},
	pdfsubject = {\enSubject},
	pdfkeywords = {\enKeywords}
}
